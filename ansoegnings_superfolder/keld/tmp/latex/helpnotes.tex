%%%%%%%%%%%%%%%%%%%%%%%%%%%%%%%%%%%%%%%%
%% THIS IS A SECTION WITH USAGE NOTES %%
%%%%%%%%%%%%%%%%%%%%%%%%%%%%%%%%%%%%%%%%

% Declare a new group to limit the scope of \color to this section.
\begingroup
\color{red}

\Section
{This is a\newline
Section\newline
With\newline
Usage Notes}
{This is a Section With Usage Notes (For PDF Bookmark)}
{PDF:ThisIsASectionWithUsageNotes:ForPDFLink}

\SubSection
{This is a SubSection}
{This is a SubSection (For PDF Bookmark)}
{PDF:ThisIsASubSection:ForPDFLink}

\Gap
\BulletItem
Use \CodeCommand{Section\{a\}\{b\}\{c\}} and
\CodeCommand{SubSection\{a\}\{b\}\{c\}}
to create sections and subsections, where
\Code{a} is the heading displayed on the page,
\Code{b} is the PDF bookmark heading, and
\Code{c} is the internal PDF link (must be unique).
Sections and subsections will appear in the PDF bookmarks.
Note the CamelCase command names.

\Gap
\BulletItem
Use
\CodeCommand{Entry},
\CodeCommand{BulletItem},
\CodeCommand{SubBulletItem},
\CodeCommand{Item},
\CodeCommand{SubItem},
\CodeCommand{NumberedItem},
etc.,
to create entries in the main body of the CV.

\Gap
\BulletItem
Enclose entry details between
\CodeCommand{begin\{Detail\}} and
\CodeCommand{end\{Detail\}}
so that they are typeset in a smaller font.
\begin{Detail}
\Item
This is an example of entry detail text enclosed in a \Code{Detail} environment.
\end{Detail}

\Gap
\BulletItem
Use \CodeCommand{Gap} and \CodeCommand{BigGap} to insert vertical spaces between entries to improve layout.

\BigGap
\SubSection
{This is Another SubSection}
{This is Another Subsection (For PDF Bookmark)}
{PDF:ThisIsAnotherSubSection:ForPDFLink}

\Gap
\Entry
This is a plain \CodeCommand{Entry},
followed by an \CodeCommand{hfill} and a date range
\hfill
\DatestampYM{2015}{10} --
\DatestampYM{2015}{12}

\Gap
\BulletItem
This is a \CodeCommand{BulletItem}.
\Item
This is an \CodeCommand{Item}, which has no bullet.
Note the alignment with the \CodeCommand{BulletItem} above.

\Gap
\SubBulletItem
This is a \CodeCommand{SubBulletItem}.
\SubItem
This is a \CodeCommand{SubItem}, which has no bullet.
Note the alignment with the \CodeCommand{SubBulletItem} above.

\Gap
\NumberedItem{[42]}
This is a \CodeCommand{NumberedItem}.
Change the value of the macro \CodeCommand{MaxNumberedItem} to adjust the indentation width.

\BigGap
\SubSection
{Line, Paragraph, and Page Breaks}
{Line, Paragraph, and Page Breaks (For PDF Bookmark)}
{PDF:LineParagraphAndPageBreaks:ForPDFLink}

\Gap
\BulletItem
To create a new line within the same paragraph (i.e., preserving the same paragraph indentation), use \CodeCommand{newline} instead of \CodeCommand{\textbackslash};
the latter will reset the paragraph indentation.

\Gap
\BulletItem
To create a new paragraph, use \CodeCommand{par} or simply leave an empty line.
Paragraph indentations (from
\CodeCommand{Entry},
\CodeCommand{BulletItem},
\CodeCommand{SubBulletItem},
\CodeCommand{Item},
\CodeCommand{SubItem},
\CodeCommand{NumberedItem},
etc.) do not carry across different paragraphs.

\Gap
\BulletItem
To create a new page, use \CodeCommand{newpage}.

\BigGap
\SubSection
{Dates}
{Dates (For PDF Bookmark)}
{PDF:Dates:ForPDFLink}

\Gap
\BulletItem
Use the following macros to specify and display dates consistently:
\SubBulletItem
\CodeCommand{DatestampYMD\{yyyy\}\{MM\}\{dd\}}
(e.g., \CodeCommand{DatestampYMD\{2008\}\{01\}\{15\}})
\SubBulletItem
\CodeCommand{DatestampYM\{yyyy\}\{MM\}}
(e.g., \CodeCommand{DatestampYM\{2008\}\{01\}})
\SubBulletItem
\CodeCommand{DatestampY\{yyyy\}}
(e.g., \CodeCommand{DatestampY\{2008\}})

\Gap
\BulletItem
Change the date format option passed to the document class to adjust how dates are displayed throughout the document:
\SubBulletItem
\Code{MMMyyyy} (``Jan~2008'')
\SubBulletItem
\Code{ddMMMyyyy} (``15~Jan~2008'')
\SubBulletItem
\Code{MMMMyyyy} (``January~2008'')
\SubBulletItem
\Code{ddMMMMyyyy} (``15~January~2008'')
\SubBulletItem
\Code{yyyyMMdd} (``2008-01-15'')
\SubBulletItem
\Code{yyyyMM} (``2008-01'')
\SubBulletItem
\Code{yyyy} (``2008'')

\endgroup